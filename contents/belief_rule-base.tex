I will begin this chapter with a less formal introduction to traditional rules and rule sets given in \ref{rulesandrulesets}.
Following it, I will go into more detail of belief rule-based systems giving a more formal presentation in \ref{TODO}.

\section{Rules and rule sets}

{\color{red}I hope this sort of informal introduction giving an intuitive introduction to the main topic to be discussed
is appropriate in a thesis. I would skip this in a journal article.}

\label{rulesandrulesets}
Rules are a natural way of conveying information on how an agent or system should work according to its' surrounding environment.
A rule can be as simple as an if... then... statement; such a rule could be for example:

\begin{displayquote}
\textbf{Rule 1:} ``\textit{If it rains outside, bring an umbrella.}``.
\end{displayquote}

%However, such rules are strict. What if Bob is not sure if it is going to rain? Should he still take an umbrella if the sky looks dark?
%Belief-rule based systems are meant to tackle such scenarios. They extend the idea of rules to include a certain degree of belief to each rule.
%Then, given an observation, such as, how dark the sky looks, a belief-rule based system can give an answer that Bob should probably bring an umbrella,
%if he decides to venture outside.
A rule consists of three elements: the precedent, the condition, and the consequent. In previous example with rule 1,
suppose there was an agent named Bob who is contemplating whether he should bring an umbrella
outside or not. The precedent in the example should give a definite answer to the question implied by the condition of
the rule "if it rains". The question implied is "Does it rain?".
Therefore, Bob needs to look outside and see if it rains or not; he needs information from the surrounding environment.
Using the precedent, the rule can be resolved: If bob observes rain, he should bring an umbrella outside.

However, the rule in the example does not state anything
regarding the case when it does not rain outside. One could argue that the rule implicitly suggests to not bring an umbrella,
but it is not an explicit consequent of the rule when it does not rain outside. To cover the case of no rain, an additional rule should be stated:

\begin{displayquote}
\textbf{Rule 2:} ``\textit{If it does not rain outside, do not bring an umbrella.}``.
\end{displayquote}

In turn, rule 2 does not state anything regarding the case when it rains outside. 
For Bob to have a coherent set of actions whether to bring an umbrella outside or not,
the rules 1 and 2 should be combined into a single rule set:

\begin{displayquote}
\textbf{Rule set 1:}
\begin{cases}
``\textit{If it rains outside, bring an umbrella.}`` \\
``\textit{If it does not rain outside, do not bring an umbrella.}`` 
\end{cases}
\end{displayquote}

Rule set 1 is now a complete rule set; a definite consequent follows all possible precedents, namely "It rains outside."
and "It does not rain outside.". If there was a third precedent in the example, for example: "It might be raining outside", rule set 1
would be incomplete and is said to contain a degree of ignorance: It does not state what to do in the case when the observation of rain
is indefinite.

However, in real life, observations are seldom definitive. To cover all possible cases of different observations made by Bob in rule set 1
would mean an addition of an indefinite number of rules with the precedents covering all possible cases: and exercise in futility.

Could the conditions of rules in a rule set have a degree of truthiness associated to them, instead of being simply true of false?
How to infer an appropriate consequent from such a rule set?
This is where belief rule-base systems come into play.

\section{Belief rules}
In a belief rule $R$, the precedent are defined as a set of finite arrays of referential values, and the consequent
is defined as a finite array of referential values. It is assumed that these values are real numbers.
A referential array set $A$, with finite arrays $A_i$, is defined for the precedent, and a finite value array $D$ is defined for the consequent.
Formally defined:

\begin{equation}
\label{A_set_def}
    A = \Set{ A_i = \Set{A_{i, n} \mid \forall n \in \left[1, \left|A_i\right| \right]}
    \mid \forall i \in \left[1, T \right] }, \quad\text{where}\quad i, n, T \in \mathbb{N};
\end{equation}

and

\begin{equation}
\label{D_set_def}
    D = \Set{D_j \mid \forall j \in \left[1, N \right] }, \quad\text{where}\quad j, N \in \mathbb{N}.
\end{equation}

In \eqref{A_set_def} and \eqref{D_set_def}, $T$ is the total number of attributes in an observation, $A_i$ is the array of possible  referential values
for the $i$th attribute in an observation,  $\left|A_i\right|$ is the number of referential values for the $i$th attribute in an observation,
and $N$ represent the total number of referential values for the consequents.

The input to a belief rule is defined as an observation $X$, which is an array of observed precedent values
for $T$ attributes defined as

\begin{equation}
    X = \Set{ x_i \mid \forall i \in \left[1, T \right]}, \quad\text{where}\quad i \in \mathbb{N}.
\end{equation}

The condition of a belief rule can be defined as a logical relationship $F$ between $X$ and $A$ as

\begin{equation}
    F : x_1\;\text{is}\;A_1\;\land\;x_2\;\text{is}\;A_2\;\land\dots\land\;x_T\;\text{is}\;A_T.
\end{equation}

Finally, a real valued belief degree $\beta_j$ can be defined for a belief rule. $\beta_j$ represents the degree of belief for $D_j$ to be the consequent
in a belief rule, when $F$ is given.

Therefore, a single belief rule can be defined as

\begin{equation}
\label{singlerule}
    \tilde{R} = \braket{X, A, D, F}{}.
\end{equation}

The single rule \eqref{singlerule} can be then be generalized to a finite set of $L \in \mathbb{N}$ belief rules as

\begin{equation}
    R = \Set{ \braket{X, A^k, D, F_k} \mid k \in \left[1, L \right]},
\end{equation}

which is understood as each rule $R_k$ having its' own set of referential value arrays $A^k$, and a logical relationship $F_k$, but the same observation $X$, and
the same consequent referential values $D$.

%For a real valued observation $X$ to be usable with a belief rule, the observation must be first expressed as a belief distribution over
%the set of referential values defined for the precedents in the rule. It is therefore assumed that $A$ is defined in a way, which allows the expression
%of the observation $X$. As presented in \ref{TODO}, the observation $X$ can be transformed to the following distribution

%\begin{equation}
%    \alpha = \Set{
%    \min{\left(
%    \max{\left( \frac{A_{i+1} - X}{A_{i+1} - A_{i}}, 0 \right)},\:
%    \max{\left( \frac{X - A_{i-1}}{A_i - A_{i-1}}, 0 \right)}
%    \right)}
%    \mid \forall i \in \left[1, T\right] },
%\end{equation}

%where $A_0 = A_T$ and $A_{T+1} = A_1$.